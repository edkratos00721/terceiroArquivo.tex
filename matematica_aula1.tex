\documentclass[12pt, a4paper, brazil, oneside]{book}
\usepackage[T1]{fontenc}
\usepackage[utf8]{inputenc}
\usepackage[brazilian]{babel}
\usepackage[top = 2cm, bottom = 2cm, left = 2.5cm, right = 2.5cm]{geometry}
\usepackage{graphicx}
\usepackage{amsmath,array, amssymb}
\linespread{1.5}



\title{Matemática}
\author{Edson Rorigues dos Santos}

\date{11 de Janeiro de 2023}

\begin{document}
\maketitle

\begin{equation}
x=2
\end{equation}
\begin{equation}
\left( \frac{a}{b}\right) 
\end{equation}
\begin{equation}
3^{2}
\end{equation}
\begin{equation}
\sqrt[3]{(4+4)}
\end{equation}
\begin{equation}
\log_{3}8  \rightarrow\cdots             
\end{equation}

\begin{equation}
\lim_{x\rightarrow 1}(x^{3}-3)
\end{equation}

\begin{equation}
\lim_{x\rightarrow 2}\sqrt{x^{4}-8}
\end{equation}

\begin{equation}
\lim_{x\rightarrow-3}\frac{x^{2}-9}{x + 3}
\end{equation}

\begin{equation}
\lim_{x\rightarrow\infty}\frac{1}{2x}
\end{equation}

\begin{equation}
\int(e^{-x} + 2^{x})dx
\end{equation}

\begin{equation}
\int_a^b f(x)dx = F(b)- F(a)
\end{equation}

\begin{equation}
\int_{a}^{b} f(x)dx = F(b) - F(a)
\end{equation}

%fisica

\begin{equation}
\vec{F}= -G\frac{m_1m_2}{r^2}\hat{r}.
\end{equation}

\begin{equation}
\ 6,6 \times  10^{-11} \frac{m^3}{Kg^{-1}s^{-2}}
\end{equation}

\begin{equation}
\left( \frac{a}{b}\right)
\left[  \frac{a}{b}\right] 
\left\lbrace \frac{a}{b}\right\rbrace  
\end{equation}

\begin{equation}
\frac{d}{dt} \left( {mr^2\frac{d\theta}{dt}}\right)= 0
\end{equation}

\begin{equation}
f(t) = \frac{1}{2} =\frac{\cos\frac{\pi}{3}}{2\pi} \sum_\infty^\infty \frac{1}{n}e^{Bn2\pi t}
\end{equation}

\begin{equation}
\left( 
\begin{array} {lr}
	a & b \\
	c & d
\end{array} \right) 
\end{equation}

\begin{equation}
\left( 
\begin{array} {lcr}
	a & b & c \\
	d & e & f \\
	g & h & i
\end{array} \right) 
\end{equation}

\begin{equation}
\begin{pmatrix}
	x&y&z\\
	w&h&r
\end{pmatrix}
\end{equation}



\end{document}