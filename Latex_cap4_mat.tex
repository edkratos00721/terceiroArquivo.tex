\documentclass[12pt, a4paper]{article}

\usepackage[T1]{fontenc}
\usepackage[utf8]{inputenc}
\usepackage[brazilian]{babel}
\usepackage[top = 2cm, bottom = 2cm,left = 2.5cm, right = 2.5cm]{geometry}
\usepackage{setspace}
\usepackage{indentfirst}

\author{Edson Rodrigues dos Santos}
\date{10 de Janeiro de 2023}
\title{Aula 04 de Matematica no LAtex}

\begin{document}
\maketitle
\begin{center}
{\large\textbf(Capitulo 4 - Ambiente matemático)}
\end{center}
\vspace{0.5cm}
\begin{flushleft}
Nome: Edson Rodrigues dos Santos \\
Date: 10 de Janeiro de 2023
\end{flushleft}
\vspace{1cm}

Aqui iremos nos adentrar no capitulo 4, que tem como tema Ambiente matemático. Segundo a equação $ x = 2 $, x está valendo 2. Abaixo irei utilzar o ambiente equation.
\begin{equation}
x = 2.
\end{equation}

\begin{equation}
\left(\frac{a}{b}\right)
\end{equation}


\end{document}